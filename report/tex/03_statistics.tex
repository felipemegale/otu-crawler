\section{Statistics of 6 networks}
In this section I will present some statistics about the 6 chosen networks. The results will be presented in tables. It is also noteworthy to reiterate that Gephi does not display nodes if their degree is 0.

\textbf{Number of nodes.} The number of nodes of each graph will be the amount of people who have at least one edge connecting them to another person.

\textbf{Number of edges.} The number of edges of each graph will be the amount of connections that exist between each person for each question.

\textbf{Edge density.} Graph density tells us how connected nodes are between each other. For undirected graphs, this metric can be calculated as
\begin{equation}
    D_{undirected} = \frac{2|E|}{|N|(|N|-1)}
    \label{equation:dir_density}
\end{equation}
and the density for directed graphs is defined as
\begin{equation}
    D_{directed} = \frac{|E|}{|N|(|N    |-1)}
    \label{equation:undir_density}
\end{equation}
where $E$ is the number of edges and $V$ is the number of nodes in the graph.

\textbf{Degree distribution.} The degree distribution of a graph allows us to grasp how deeply connected the nodes are. Figure \ref{fig:4} pictures the degree distributions for the 6 chosen networks.
\begin{figure}
    \centering
    \subfloat[\centering Network 1]{{\includegraphics[width=0.45\textwidth]{img/net_0_degree_distribution.png}}}
    \qquad
    \subfloat[\centering Network 2]{{\includegraphics[width=0.45\textwidth]{img/net_1_degree_distribution.png}}}
    \qquad
    \subfloat[\centering Network 3]{{\includegraphics[width=0.45\textwidth]{img/net_2_degree_distribution.png}}}
    \qquad
    \subfloat[\centering Network 4]{{\includegraphics[width=0.45\textwidth]{img/net_3_degree_distribution.png}}}
    \qquad
    \subfloat[\centering Network 5]{{\includegraphics[width=0.45\textwidth]{img/net_4_degree_distribution.png}}}
    \qquad
    \subfloat[\centering Network 7]{{\includegraphics[width=0.45\textwidth]{img/net_6_degree_distribution.png}}}
    \caption{Degree distributions}
    \label{fig:4}
\end{figure}

\textbf{Average clustering coefficient.} The average clustering coefficient for a graph helps determine how transitive a relationship is. The clustering coefficient is defined as
\begin{equation}
    C_i = \frac{2e_i}{k_i(k_i-1)}
    \label{equation:clustering_coef}
\end{equation}
where $e_i$ is the number of edges between the neighbors of node $i$.

The average clustering coefficient of the graph is calculated as
\begin{equation}
    <C> = \frac{1}{N}\sum_{i}^{N}C_i
    \label{equation:avg_clustering_coef}
\end{equation}
where $N$ is the number of nodes in the graph, and $C_i$ is the clustering coefficient of node $i$.

\textbf{Number of nodes in strongly connected component (SCC).} The strongly connected component (SCC) metric can only be obtained from directed graphs. Since only the first network, then it is the only one that can provide this value. For networks 2 through 5, and 7, the values are from the connected components. Refer to table \ref{table:1} for the values.

\textbf{Number of nodes in weakly connected component (WCC).} The weakly connected component (WCC) metric can only be obtained from directed graphs. Since only the first network, then it is the only one that can provide this value. For networks 2 through 5, and 7, the values are from the connected components. Refer to table \ref{table:1} for the values.

\textbf{Average path length in SCC.}

\textbf{Diameter of SCC.}

\textbf{Community detection.}

\begin{table}
    \centering
    \begin{tabular}{|c|c|c|c|c|c|c|c|}
        \hline
        \textbf{Metric} & \textbf{Net. 1} & \textbf{Net. 2} & \textbf{Net. 3} & \textbf{Net. 4} & \textbf{Net. 5} & \textbf{Net. 7} \\
        \hline
        Num. Nodes & 59 & 55 & 52 & 41 & 29 & 59 \\
        \hline
        Num. Edges & 427 & 124 & 80 & 54 & 34 & 216 \\
        \hline
        Density & 0.125 & 0.084 & 0.060 & 0.066 & 0.084 & 0.126 \\
        \hline
        Avg. Clust. Coef. & 0.375 & 0.412 & 0.482 & 0.583 & 0.620 & 0.542 \\
        \hline
        Num. Nodes SCC & 49 & - & - & - & - & - \\
        \hline
        Num. Nodes WCC & 58 & - & - & - & - & - \\
        \hline
        Num. Nodes CC & - & 52 & 42 & 20 & 10 & 59 \\
        \hline
        Avg. Path Len. SCC &  &  &  &  &  &  \\
        \hline
        Diameter of SCC &  &  &  &  &  &  \\
        \hline
        Comm. Detection &  &  &  &  &  &  \\
        \hline
    \end{tabular}
    \caption{Network statistics}
    \label{table:1}
\end{table}
