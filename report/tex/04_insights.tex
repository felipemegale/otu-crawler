\section{Insights}
All the metrics previously calculated and visualized allow us to extract insightful knowledge about the networks that exist in our graduate program. Tackling the aspects of low degree distribution, low density and high number of communities, we can see that the questions \textit{"Which person you have collaborated with?"}, \textit{"Which person you have eye contact?"}, and \textit{"Which peson you have eaten lunch with?"} create the most sparse graphs. That may be due to the fact that most classes are still online and people haven't gotten the chance to interact with each other in a more meaningful way.

It is also interesting to notice unexpected behaviors on these sparse networks. Let us look at network 3 for example. It was one the three networks that were low in density and very sparse. However, the existing connected component was well connected. This means that despite existing many people who did not collaborate with their peers, the ones who did were thorough in doing it. However, because there were few edges in the whole graph, the average path length in the connected component was high. This is an indication that people collaborate together but they have their usual partners.

For network 5, which asks the question \textit{"Which peson you have eaten lunch with?"}, we see that not many people have participated in this activity. Those who did, usually do frequently and there seems to be a closed group. Visualizing the network we are able to see that the connected component is almost a complete graph. This means that even though there are few connections in the graph, the ones that exist are meaningful.

Finally, discovering which nodes in each graph are more central is important. The reason for this is that knowing which people connect two or more components may contribute to how the people in these networks interact and collaborate with each other. Removing some of the key people can result in disrupting an intricate collaboration network.